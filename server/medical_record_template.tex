%-------------------------
% Medical Record Template in LaTeX
% Based on Healio documentation requirements
%------------------------

\documentclass[letterpaper,11pt]{article}

\usepackage{latexsym}
\usepackage[empty]{fullpage}
\usepackage{titlesec}
\usepackage{marvosym}
\usepackage[usenames,dvipsnames]{color}
\usepackage{verbatim}
\usepackage{enumitem}
\usepackage[hidelinks]{hyperref}
\usepackage{fancyhdr}
\usepackage[english]{babel}
\usepackage{tabularx}
\usepackage{booktabs}
\usepackage{ragged2e}
\usepackage{xcolor}
\usepackage{fontawesome}
\usepackage{geometry}
\usepackage{array}
\usepackage{graphicx}
\usepackage{etoolbox}

% Set page geometry
\geometry{
  letterpaper,
  top=0.75in,
  bottom=0.75in,
  left=0.75in,
  right=0.75in
}

% Page style
\pagestyle{fancy}
\fancyhf{} % clear all header and footer fields
\renewcommand{\headrulewidth}{0pt}
\renewcommand{\footrulewidth}{0pt}
\fancyfoot[C]{\thepage}

% Define colors
\definecolor{medblue}{RGB}{0, 84, 147} % Medical blue color
\definecolor{medgray}{RGB}{100, 100, 100} % Subtle gray for secondary text

% Format section titles
\titleformat{\section}{
  \vspace{-4pt}\scshape\raggedright\large\color{medblue}
}{}{0em}{}[\color{medblue}\titlerule \vspace{8pt}]

\titleformat{\subsection}{
  \vspace{-2pt}\scshape\raggedright\normalsize\color{medblue}
}{}{0em}{}[\vspace{-2pt}]

% Custom commands for different formats
\newcommand{\patientHeader}[5]{
  \begin{center}
    \textbf{\Large \scshape Medical Record} \\
    \vspace{10pt}
    \begin{tabular}{r l r l}
      \textbf{Patient Name:} & #1 & \textbf{Date:} & #2 \\
      \textbf{DOB:} & #3 & \textbf{MRN:} & #4 \\
      \textbf{Provider:} & #5 & & \\
    \end{tabular}
  \end{center}
  \vspace{0.8cm}
}

% Command for paragraph format
\newcommand{\paragraphSection}[2]{
  \section{#1}
  \vspace{-0.2cm}
  \begin{justify}
    #2
  \end{justify}
  \vspace{0.4cm}
}

% Command for bullet points
\newcommand{\bulletSection}[2]{
  \section{#1}
  \vspace{-0.2cm}
  \begin{itemize}[leftmargin=0.5cm, label={\small\textbullet}, itemsep=0pt]
    #2
  \end{itemize}
  \vspace{0.4cm}
}

% Command for numbered bullets
\newcommand{\numberedSection}[2]{
  \section{#1}
  \vspace{-0.2cm}
  \begin{enumerate}[leftmargin=0.7cm, label=\arabic*., itemsep=0pt]
    #2
  \end{enumerate}
  \vspace{0.4cm}
}

% Command for vital signs table
\newcommand{\vitalSignsSection}[5]{
  \section{Vital Signs}
  \vspace{-0.3cm}
  \begin{center}
    \begin{tabular}{|l|c|}
      \hline
      \textbf{Blood Pressure} & #1 \\
      \hline
      \textbf{Pulse} & #2 \\
      \hline
      \textbf{Respiratory Rate} & #3 \\
      \hline
      \textbf{Temperature} & #4 \\
      \hline
      \textbf{Oxygen Saturation} & #5 \\
      \hline
    \end{tabular}
  \end{center}
  \vspace{0.3cm}
}

% For generic vital signs with variable number of entries
\newcommand{\genericVitalSignsSection}[1]{
  \section{Vital Signs}
  \vspace{-0.3cm}
  \begin{itemize}[leftmargin=0.5cm, label={\small\textbullet}, itemsep=0pt]
    #1
  \end{itemize}
  \vspace{0.3cm}
}

% Helper command for bullet items
\newcommand{\bulletItem}[1]{
  \item #1
}

% Helper command for numbered items
\newcommand{\numberedItem}[1]{
  \item #1
}

\begin{document}

% To be populated from patient data
\patientHeader{[PATIENT_NAME]}{[DATE]}{[DOB]}{[MRN]}{[PROVIDER]}

% Chief Complaint Section (Numbered Bullets)
\numberedSection{Chief Complaint}{
  % To be populated from JSON: chief_complaint.items
  % Example: \numberedItem{Progressive dyspnea on exertion.}
  [CHIEF_COMPLAINT_ITEMS]
}

% History Section (Paragraph)
\paragraphSection{History of Present Illness}{
  % To be populated from JSON: history.content
  [HISTORY_CONTENT]
}

% Subjective Section (Paragraph)
\paragraphSection{Subjective}{
  % To be populated from JSON: subjective.content
  [SUBJECTIVE_CONTENT]
}

% Vital Signs Section (Bullet Points or Table)
% Use \vitalSignsSection{BP}{Pulse}{Resp}{Temp}{O2} for standard vital signs
% Or use \genericVitalSignsSection{...} for bullet point format
\genericVitalSignsSection{
  % To be populated from JSON: vital_signs.items
  % Example: \bulletItem{Blood pressure: 140/90 mmHg}
  [VITAL_SIGNS_ITEMS]
}

% Physical Examination Section (Bullet Points)
\bulletSection{Physical Examination}{
  % To be populated from JSON: physical.items
  % Example: \bulletItem{Patient appears well-nourished and in no acute distress.}
  [PHYSICAL_ITEMS]
}

% Objective Section (Bullet Points)
\bulletSection{Objective}{
  % To be populated from JSON: objective.items
  % Example: \bulletItem{Blood pressure measured at 140/90 mmHg.}
  [OBJECTIVE_ITEMS]
}

% Assessment Section (Numbered Bullets)
\numberedSection{Assessment}{
  % To be populated from JSON: assessment.items
  % Example: \numberedItem{Stable angina pectoris.}
  [ASSESSMENT_ITEMS]
}

% Plan Section (Numbered Bullets)
\numberedSection{Plan}{
  % To be populated from JSON: plan.items
  % Example: \numberedItem{Stress echocardiogram to evaluate for myocardial ischemia.}
  [PLAN_ITEMS]
}

% Medications Section (Numbered Bullets)
\numberedSection{Medications}{
  % To be populated from JSON: medications.items
  % Example: \numberedItem{Atorvastatin 20 mg daily for cholesterol management.}
  [MEDICATIONS_ITEMS]
}

% Notes Section (Paragraph)
\paragraphSection{Additional Notes}{
  % To be populated from JSON: notes.content
  [NOTES_CONTENT]
}

\end{document}